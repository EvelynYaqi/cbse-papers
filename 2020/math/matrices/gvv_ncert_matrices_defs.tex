\let\negmedspace\undefined
\let\negthickspace\undefined
%\RequirePackage{amsmath}
\documentclass[journal,12pt,twocolumn]{IEEEtran}
%
% \usepackage{setspace}
 \usepackage{gensymb}
%\doublespacing
%\singlespacing
%\usepackage{silence}
%Disable all warnings issued by latex starting with "You have..."
%\usepackage{graphicx}
\usepackage{amssymb}
%\usepackage{relsize}
\usepackage[cmex10]{amsmath}
%\usepackage{amsthm}
%\interdisplaylinepenalty=2500
%\savesymbol{iint}
%\usepackage{txfonts}
%\restoresymbol{TXF}{iint}
%\usepackage{wasysym}
\usepackage{amsthm}
%\usepackage{pifont}
%\usepackage{iithtlc}
% \usepackage{mathrsfs}
% \usepackage{txfonts}
 \usepackage{stfloats}
% \usepackage{steinmetz}
 \usepackage{bm}
% \usepackage{cite}
% \usepackage{cases}
% \usepackage{subfig}
%\usepackage{xtab}
\usepackage{longtable}
%\usepackage{multirow}
%\usepackage{algorithm}
%\usepackage{algpseudocode}
\usepackage{enumitem}
 \usepackage{mathtools}
 \usepackage{tikz}
% \usepackage{circuitikz}
% \usepackage{verbatim}
%\usepackage{tfrupee}
\usepackage[breaklinks=true]{hyperref}
%\usepackage{stmaryrd}
%\usepackage{tkz-euclide} % loads  TikZ and tkz-base
%\usetkzobj{all}
\usepackage{listings}
    \usepackage{color}                                            %%
    \usepackage{array}                                            %%
    \usepackage{longtable}                                        %%
    \usepackage{calc}                                             %%
    \usepackage{multirow}                                         %%
    \usepackage{hhline}                                           %%
    \usepackage{ifthen}                                           %%
  %optionally (for landscape tables embedded in another document): %%
    \usepackage{lscape}     
% \usepackage{multicol}
% \usepackage{chngcntr}
%\usepackage{enumerate}

%\usepackage{wasysym}
%\newcounter{MYtempeqncnt}
\DeclareMathOperator*{\Res}{Res}
\DeclareMathOperator*{\equals}{=}
%\renewcommand{\baselinestretch}{2}
\renewcommand\thesection{\arabic{section}}
\renewcommand\thesubsection{\thesection.\arabic{subsection}}
\renewcommand\thesubsubsection{\thesubsection.\arabic{subsubsection}}

\renewcommand\thesectiondis{\arabic{section}}
\renewcommand\thesubsectiondis{\thesectiondis.\arabic{subsection}}
\renewcommand\thesubsubsectiondis{\thesubsectiondis.\arabic{subsubsection}}

% correct bad hyphenation here
\hyphenation{op-tical net-works semi-conduc-tor}
\def\inputGnumericTable{}                                 %%

\lstset{
%language=C,
frame=single, 
breaklines=true,
columns=fullflexible
}
%\lstset{
%language=tex,
%frame=single, 
%breaklines=true
%}
\begin{document}

%


\newtheorem{theorem}{Theorem}[section]
\newtheorem{problem}{Problem}
\newtheorem{proposition}{Proposition}[section]
\newtheorem{lemma}{Lemma}[section]
\newtheorem{corollary}[theorem]{Corollary}
\newtheorem{example}{Example}[section]
\newtheorem{definition}[problem]{Definition}
%\newtheorem{thm}{Theorem}[section] 
%\newtheorem{defn}[thm]{Definition}
%\newtheorem{algorithm}{Algorithm}[section]
%\newtheorem{cor}{Corollary}
\newcommand{\BEQA}{\begin{eqnarray}}
\newcommand{\EEQA}{\end{eqnarray}}
\newcommand{\define}{\stackrel{\triangle}{=}}
\newcommand*\circled[1]{\tikz[baseline=(char.base)]{
    \node[shape=circle,draw,inner sep=2pt] (char) {#1};}}
\bibliographystyle{IEEEtran}
%\bibliographystyle{ieeetr}


\providecommand{\mbf}{\mathbf}
\providecommand{\pr}[1]{\ensuremath{\Pr\left(#1\right)}}
\providecommand{\qfunc}[1]{\ensuremath{Q\left(#1\right)}}
\providecommand{\sbrak}[1]{\ensuremath{{}\left[#1\right]}}
\providecommand{\lsbrak}[1]{\ensuremath{{}\left[#1\right.}}
\providecommand{\rsbrak}[1]{\ensuremath{{}\left.#1\right]}}
\providecommand{\brak}[1]{\ensuremath{\left(#1\right)}}
\providecommand{\lbrak}[1]{\ensuremath{\left(#1\right.}}
\providecommand{\rbrak}[1]{\ensuremath{\left.#1\right)}}
\providecommand{\cbrak}[1]{\ensuremath{\left\{#1\right\}}}
\providecommand{\lcbrak}[1]{\ensuremath{\left\{#1\right.}}
\providecommand{\rcbrak}[1]{\ensuremath{\left.#1\right\}}}
\theoremstyle{remark}
\newtheorem{rem}{Remark}
\newcommand{\sgn}{\mathop{\mathrm{sgn}}}
\providecommand{\abs}[1]{\left\vert#1\right\vert}
\providecommand{\res}[1]{\Res\displaylimits_{#1}} 
\providecommand{\norm}[1]{\left\lVert#1\right\rVert}
%\providecommand{\norm}[1]{\lVert#1\rVert}
\providecommand{\mtx}[1]{\mathbf{#1}}
\providecommand{\mean}[1]{E\left[ #1 \right]}
\providecommand{\fourier}{\overset{\mathcal{F}}{ \rightleftharpoons}}
%\providecommand{\hilbert}{\overset{\mathcal{H}}{ \rightleftharpoons}}
\providecommand{\system}{\overset{\mathcal{H}}{ \longleftrightarrow}}
	%\newcommand{\solution}[2]{\textbf{Solution:}{#1}}
\newcommand{\solution}{\noindent \textbf{Solution: }}
\newcommand{\cosec}{\,\text{cosec}\,}
\providecommand{\dec}[2]{\ensuremath{\overset{#1}{\underset{#2}{\gtrless}}}}
\newcommand{\myvec}[1]{\ensuremath{\begin{pmatrix}#1\end{pmatrix}}}
\newcommand{\mydet}[1]{\ensuremath{\begin{vmatrix}#1\end{vmatrix}}}
%\numberwithin{equation}{section}
\numberwithin{equation}{subsection}
%\numberwithin{problem}{section}
%\numberwithin{definition}{section}
\makeatletter
\@addtoreset{figure}{problem}
\makeatother

\let\StandardTheFigure\thefigure
\let\vec\mathbf
%\renewcommand{\thefigure}{\theproblem.\arabic{figure}}
\renewcommand{\thefigure}{\theproblem}
%\setlist[enumerate,1]{before=\renewcommand\theequation{\theenumi.\arabic{equation}}
%\counterwithin{equation}{enumi}


%\renewcommand{\theequation}{\arabic{subsection}.\arabic{equation}}

\def\putbox#1#2#3{\makebox[0in][l]{\makebox[#1][l]{}\raisebox{\baselineskip}[0in][0in]{\raisebox{#2}[0in][0in]{#3}}}}
     \def\rightbox#1{\makebox[0in][r]{#1}}
     \def\centbox#1{\makebox[0in]{#1}}
     \def\topbox#1{\raisebox{-\baselineskip}[0in][0in]{#1}}
     \def\midbox#1{\raisebox{-0.5\baselineskip}[0in][0in]{#1}}

\vspace{3cm}

\title{
	%\logo{
%Computational Approach to School Geometry
	Matrix Analysis
%	}
}
\author{ G V V Sharma$^{*}$% <-this % stops a space
	\thanks{*The author is with the Department
		of Electrical Engineering, Indian Institute of Technology, Hyderabad
		502285 India e-mail:  gadepall@iith.ac.in. All content in this manual is released under GNU GPL.  Free and open source.}
	
}	
%\title{
%	\logo{Matrix Analysis through Octave}{\begin{center}\includegraphics[scale=.24]{tlc}\end{center}}{}{HAMDSP}
%}


% paper title
% can use linebreaks \\ within to get better formatting as desired
%\title{Matrix Analysis through Octave}
%
%
% author names and IEEE memberships
% note positions of commas and nonbreaking spaces ( ~ ) LaTeX will not break
% a structure at a ~ so this keeps an author's name from being broken across
% two lines.
% use \thanks{} to gain access to the first footnote area
% a separate \thanks must be used for each paragraph as LaTeX2e's \thanks
% was not built to handle multiple paragraphs
%

%\author{<-this % stops a space
%\thanks{}}
%}
% note the % following the last \IEEEmembership and also \thanks - 
% these prevent an unwanted space from occurring between the last author name
% and the end of the author line. i.e., if you had this:
% 
% \author{....lastname \thanks{...} \thanks{...} }
%                     ^------------^------------^----Do not want these spaces!
%
% a space would be appended to the last name and could cause every name on that
% line to be shifted left slightly. This is one of those "LaTeX things". For
% instance, "\textbf{A} \textbf{B}" will typeset as "A B" not "AB". To get
% "AB" then you have to do: "\textbf{A}\textbf{B}"
% \thanks is no different in this regard, so shield the last } of each \thanks
% that ends a line with a % and do not let a space in before the next \thanks.
% Spaces after \IEEEmembership other than the last one are OK (and needed) as
% you are supposed to have spaces between the names. For what it is worth,
% this is a minor point as most people would not even notice if the said evil
% space somehow managed to creep in.

%\WarningFilter{latex}{LaTeX Warning: You have requested, on input line 117, version}


% The paper headers
%\markboth{Journal of \LaTeX\ Class Files,~Vol.~6, No.~1, January~2007}%
%{Shell \MakeLowercase{\textit{et al.}}: Bare Demo of IEEEtran.cls for Journals}
% The only time the second header will appear is for the odd numbered pages
% after the title page when using the twoside option.
% 
% *** Note that you probably will NOT want to include the author's ***
% *** name in the headers of peer review papers.                   ***
% You can use \ifCLASSOPTIONpeerreview for conditional compilation here if
% you desire.




% If you want to put a publisher's ID mark on the page you can do it like
% this:
%\IEEEpubid{0000--0000/00\$00.00~\copyright~2007 IEEE}
% Remember, if you use this you must call \IEEEpubidadjcol in the second
% column for its text to clear the IEEEpubid mark.



% make the title area
\maketitle

\newpage

\tableofcontents

\bigskip

\renewcommand{\thefigure}{\theenumi}
\renewcommand{\thetable}{\theenumi}
%\renewcommand{\theequation}{\theenumi}

%\begin{abstract}
%%\boldmath
%In this letter, an algorithm for evaluating the exact analytical bit error rate  (BER)  for the piecewise linear (PL) combiner for  multiple relays is presented. Previous results were available only for upto three relays. The algorithm is unique in the sense that  the actual mathematical expressions, that are prohibitively large, need not be explicitly obtained. The diversity gain due to multiple relays is shown through plots of the analytical BER, well supported by simulations. 
%
%\end{abstract}
% IEEEtran.cls defaults to using nonbold math in the Abstract.
% This preserves the distinction between vectors and scalars. However,
% if the journal you are submitting to favors bold math in the abstract,
% then you can use LaTeX's standard command \boldmath at the very start
% of the abstract to achieve this. Many IEEE journals frown on math
% in the abstract anyway.

% Note that keywords are not normally used for peerreview papers.
%\begin{IEEEkeywords}
%Cooperative diversity, decode and forward, piecewise linear
%\end{IEEEkeywords}



% For peer review papers, you can put extra information on the cover
% page as needed:
% \ifCLASSOPTIONpeerreview
% \begin{center} \bfseries EDICS Category: 3-BBND \end{center}
% \fi
%
% For peerreview papers, this IEEEtran command inserts a page break and
% creates the second title. It will be ignored for other modes.
%\IEEEpeerreviewmaketitle

\begin{abstract}
This manual provides an introduction to vectors and their properties,  based on the question papers, year 2020,  from Class 10 and 12, CBSE; JEE and JNTU.  
\end{abstract}

\section{Definitions}
\subsection{$2\times 1$ vectors}
\renewcommand{\theequation}{\theenumi}
%\begin{enumerate}[label=\arabic*.,ref=\theenumi]
\begin{enumerate}[label=\thesubsection.\arabic*.,ref=\thesubsection.\theenumi]
\numberwithin{equation}{enumi}
\item Let 
\begin{align}
  \vec{A} \equiv \overrightarrow{A} &= \myvec{a_1\\a_2} 
  \\
  &\equiv a_1\overrightarrow{i}+a_2\overrightarrow{j}, 
  \\
  \vec{B} &= \myvec{b_1\\b_2}, 
\end{align}
be $2 \times 1$ vectors.
Then, the determinant of the $2 \times 2$ matrix 
\begin{align}  
  \vec{M} = \myvec{\vec{A} & \vec{B}}
\end{align}
is defined as
\begin{align}
  \label{eq:det2d}
  \mydet{\vec{M}} &= \mydet{\vec{A} & \vec{B}} 
  \\
  &= \mydet{a_1 & b_1\\a_2 & b_2} = a_1b_2 - a_2 b_1
\end{align}
%
\item The value of the cross product of two vectors is given by  
  \eqref{eq:det2d}.
\item The area of the triangle with vertices $\vec{A}, \vec{B}, \vec{C}$ is given by the absolute value of 
\begin{align}
  \label{eq:area2d}
\frac{1}{2} \mydet{\vec{A-B} & \vec{A-C}}
  \end{align}
  \item  The transpose of $\vec{A}$ is defined as
\begin{align}
  \label{eq:transpose2d}
  \vec{A}^{\top}  = \myvec{a_1 & a_2}
\end{align}
%
\item The {\em inner product} or {\em dot product} is defined as
\begin{align}
  \label{eq:dot2d}
  \vec{A}^{\top} \vec{B} &\equiv \vec{A} \cdot \vec{B} 
  \\
  &= \myvec{a_1 & a_2} \myvec{b_1 \\ b_2}= a_1b_1+a_2b_2 
\end{align}
%
\item {\em norm} of $\vec{A}$ is defined as
\begin{align}
  \label{eq:norm2d}
  \norm{A} &\equiv \mydet{\overrightarrow{A}}
  \\
  &= \sqrt{\vec{A}^{\top} \vec{A}}= \sqrt{a_1^2+a_2^2}
\end{align}
Thus, 
\begin{align}
  \label{eq:norm2d_const}
  \norm{\lambda \vec{A}} &\equiv \mydet{\lambda\overrightarrow{A}}
  \\
  &= \abs{\lambda} \norm{\vec{A}}
\end{align}
\item The distance betwen the points $\vec{A}$ and $\vec{B}$ is given by 
\begin{align}
  \label{eq:norm2d_dist}
\norm{\vec{A}-\vec{B}} 
\end{align}
\item Let $\vec{x}$ be equidistant from the points $\vec{A}$ and $\vec{B}$.  Then 
  \begin{align}
	  \brak{\vec{A}-\vec{B}}^{\top}{\vec{x}} 
	  =  \frac{\norm{\vec{A}}^2 - \norm{\vec{B}}^2}{2}
  \label{eq:norm2d_equidist}
  \end{align}
  \solution 
\begin{align}
	\norm{\vec{x}-\vec{A}} &=
\norm{\vec{A}-\vec{B}} 
\\
	\implies \norm{\vec{x}-\vec{A}}^2 &=
\norm{\vec{x}-\vec{B}}^2 
\end{align}
which can be expressed as 
\begin{multline}
%  \label{eq:norm2d_dist}
	\brak{\vec{x}-\vec{A}}^{\top} \brak{\vec{x}-\vec{A}}=
	\brak{\vec{x}-\vec{B}}^{\top} 
\brak{\vec{x}-\vec{B}}
\\
	\implies	\norm{\vec{x}}^2-2{\vec{x}}^{\top}\vec{A} + \norm{\vec{A}}^2
	\\= \norm{\vec{x}}^2-2{\vec{x}}^{\top}\vec{B} + \norm{\vec{B}}^2
\end{multline}
which can be simplified to obtain
  \eqref{eq:norm2d_equidist}.
\item If $\vec{x}$ lies on the  $x$-axis and is  equidistant from the points $\vec{A}$ and $\vec{B}$, 
  \begin{align}
	  \vec{x} &=
	   x\vec{e}_1
  \end{align}
  where 
  \begin{align}
	  x &=\frac{\norm{\vec{A}}^2 -\norm{\vec{B}}^2 }{2\brak{\vec{A}-\vec{B}}^{\top }\vec{e}_1
}
	  \label{eq:cbse_10_x}
  \end{align}
  \solution 
  From \eqref{eq:norm2d_equidist}.
  \begin{align}
	   x\brak{\vec{A}-\vec{B}}^{\top }\vec{e}_1
		  &=
	  \frac{\norm{\vec{A}}^2 -\norm{\vec{B}}^2 }{2}
   \end{align}
	  yielding \eqref{eq:cbse_10_x}.
  \item The angle between two vectors is given by 
  \begin{align}
    \label{eq:angle2d}
    \theta = \cos^{-1}\frac{\vec{A}^{\top} \vec{B}}{\norm{A}\norm{B}}
  \end{align}
  \item If two vectors are orthogonal (perpendicular), 
  \begin{align}
    \label{eq:angle2d_orth}
\vec{A}^{\top} \vec{B} = 0
  \end{align}

  \item The {\em direction vector} of the line joining two points $\vec{A},\vec{B}$ is given by 
  \begin{align}
    \label{eq:dir_vec}
    \vec{m} = \vec{A}-\vec{B}
  \end{align}
\item The unit vector in the direction of $\vec{m}$ is defined as
\begin{align}
    \frac{\vec{m}}{\norm{\vec{m}}}
\end{align}
\item If the direction vector of a line is expressed as 
	\begin{align}
    \vec{m} = \myvec{1\\m},
\end{align}
 the $m$ is defined to be the {\em} slope of the line. 
  \item The {\em normal vector} to $\vec{m}$ is defined by 
  \begin{align}
    \label{eq:normal_vec}
    \vec{m}^{\top}  \vec{n} = 0
  \end{align}
  \item The point $\vec{P}$ that divides the line segment $AB$ in the ratio $k:1$  is given by 

  \begin{align}
	  \vec{P}&= \frac{k\vec{B}+ \vec{A}}{k+1}
	  \label{eq:section_formula}
  \end{align}
\item  The standard basis vectors are defined as 

  \begin{align}
  \vec{e}_1&= \myvec{1\\0}, 
  \vec{e}_2&= \myvec{0\\1} 
  \end{align}
\end{enumerate}
\subsection{$3\times 1$ vectors}
\renewcommand{\theequation}{\theenumi}
%\begin{enumerate}[label=\arabic*.,ref=\theenumi]
\begin{enumerate}[label=\thesubsection.\arabic*.,ref=\thesubsection.\theenumi]
\numberwithin{equation}{enumi}

\item Let 
\begin{align}
  \vec{A} &= \myvec{a_1\\a_2 \\ a_3} \equiv a_1\overrightarrow{i}+a_2\overrightarrow{j}+a_3\overrightarrow{j}, 
  \\
  \vec{B} &= \myvec{b_1\\b_2 \\ b_3}, 
\end{align}
and 
\begin{align}
  \vec{A}_{ij} &= \myvec{a_i\\a_j}, 
  \vec{B}_{ij} &= \myvec{b_i\\b_j}, 
\end{align}

\item The {\em cross product} or {\em vector product} of $\vec{A}, \vec{B}$ is defined as
\begin{align}
  \label{eq:cross3d}
	\vec{A} \times \vec{B} = \myvec{ \mydet{\vec{A}_{23} & \vec{B}_{23}} \\ \mydet{\vec{A}_{31} & \vec{B}_{31}} \\ \mydet{\vec{A}_{12}  & \vec{B}_{12}}}
\end{align}
\item Verify that
\begin{align}
  \vec{A} \times \vec{B} = -  \vec{B} \times \vec{A} 
\end{align}
\item The area of a triangle is given by 
\begin{align}
	\frac{1}{2} \norm{  \vec{A} \times \vec{B}}
\end{align}
\end{enumerate}
\subsection{Eigenvalues and Eigenvectors}
\renewcommand{\theequation}{\theenumi}
%\begin{enumerate}[label=\arabic*.,ref=\theenumi]
\begin{enumerate}[label=\thesubsection.\arabic*.,ref=\thesubsection.\theenumi]
\numberwithin{equation}{enumi}
\item The eigenvalue $\lambda$ and the eigenvector $\vec{x}$  for a matrix $\vec{A}$ are defined as, 
\begin{align}
  \vec{A} \vec{x} = \lambda \vec{x}
\end{align}
\item The eigenvalues are calculated by solving the
equation
\begin{align}
  \label{eq:chareq}
f\brak{\lambda} = \mydet{\lambda \vec{I}- \vec{A} } =0
\end{align}
The above equation is known as the characteristic equation.
\item According to the Cayley-Hamilton theorem,
\begin{align}
	\label{eq:cayley}
  f(\lambda) = 0 \implies f\brak{\vec{A}} = 0
\end{align}
\item The trace of a square  matrix is defined to be the sum of the diagonal elements.
\begin{align}
	\label{eq:trace}
	\text{tr}\brak{\vec{A}}=\sum_{i=1}^{N}a_{ii}.
\end{align}
	where $a_{ii}$ is the $i$th diagonal element of the matrix $\vec{A}$. 	
\item The trace of a matrix is equal to the sum of the eigenvalues
\begin{align}
	\label{eq:trace_eig}
	\text{tr}\brak{\vec{A}}=\sum_{i=1}^{N}\lambda_i
\end{align}


\end{enumerate}
\subsection{Determinants}
\renewcommand{\theequation}{\theenumi}
%\begin{enumerate}[label=\arabic*.,ref=\theenumi]
\begin{enumerate}[label=\thesubsection.\arabic*.,ref=\thesubsection.\theenumi]
\numberwithin{equation}{enumi}

\item Let 
\begin{align}
	\vec{A} = \myvec{a_1 & b_1 & c_1  \\ a_2 & b_2 & c_2  \\ a_3 & b_3 & c_3}.
\end{align}
be a $3 \times 3$ matrix. 
Then, 
\begin{multline}
	\mydet{\vec{A}} = a_1 \myvec{ b_2 & c_2 \\  b_3 & c_3} - a_2\myvec{ b_1 & c_1 \\  b_3 & c_3 }  \\ + a_3\myvec{a_1 & b_1 \\ a_2 & b_2 }.
\end{multline}
\item Let $\lambda_1,\lambda_2, \dots, \lambda_n$ be the eigenvalues of a matrix $\vec{A}$.  Then,   the product of the eigenvalues is equal to the determinant of $\vec{A}$.
\begin{align}
	\mydet{\vec{A}} = \prod_{i=1}^{n}\lambda_i
\end{align}
%
\item 
\begin{align}
	\mydet{\vec{A}\vec{B}} = \mydet{\vec{A}}\mydet{\vec{B}}
\end{align}
\item If $\vec{A}$ be an $n \times n$ matrix, 
\begin{align}
	\label{eq:det_kord}
	\mydet{k\vec{A}} = k^n\mydet{\vec{A}}
\end{align}

\end{enumerate}
\subsection{Rank of a Matrix}
\renewcommand{\theequation}{\theenumi}
%\begin{enumerate}[label=\arabic*.,ref=\theenumi]
\begin{enumerate}[label=\thesubsection.\arabic*.,ref=\thesubsection.\theenumi]
\numberwithin{equation}{enumi}
\item The rank of a matrix is defined as the number of linearly independent rows.  This is also known as the row rank.
\item Row rank = Column rank.
\item The rank of a matrix is obtained as the number of nonzero rows obtained after row reduction.
\item An $n \times n$ matrix is invertible if and only if its rank is $n$.
\end{enumerate}
\subsection{Inverse of a Matrix}
\renewcommand{\theequation}{\theenumi}
%\begin{enumerate}[label=\arabic*.,ref=\theenumi]
\begin{enumerate}[label=\thesubsection.\arabic*.,ref=\thesubsection.\theenumi]
\numberwithin{equation}{enumi}
\item For a $2 \times 2$ matrix 
\begin{align}
	\vec{A} = \myvec{a_1 & b_1  \\ a_2 & b_2 },
\end{align}
the inverse is given by 
\begin{align}
	\vec{A}^{-1} = \frac{1}{\mydet{\vec{A}}}\myvec{b_2 & -b_1  \\ -a_2 & a_1 },
\end{align}
\item For higher order matrices, the inverse should be calculated using row operations.
\end{enumerate}
\section{Linear Forms}
\subsection{Two Dimensions}
\renewcommand{\theequation}{\theenumi}
%\begin{enumerate}[label=\arabic*.,ref=\theenumi]
\begin{enumerate}[label=\thesubsection.\arabic*.,ref=\thesubsection.\theenumi]
\numberwithin{equation}{enumi}
\item The equation of a line  is given by  
\begin{align}
	\label{eq:normal_line}
   \vec{n}^{\top}\vec{x} = c
\end{align}
		where $\vec{n}$ is the normal vector of the line.
	\item The equation of a line with normal vector $\vec{n}$ and passing through a point $\vec{A}$ 
		is given by 
\begin{align}
	\label{eq:normal_line_pt}
	\vec{n}^{\top}\brak{\vec{x}-\vec{A}} =0 
\end{align}
\item The parametric equation of a line  is given by  
\begin{align}
	\label{eq:dir_line}
	\vec{x} = \vec{A} + \lambda \vec{m}
\end{align}
		where $\vec{m}$ is the direction vector of the line and $\vec{A}$ is any point on the line.
	\item The distance from a point $\vec{P}$ to the line  in 
	\eqref{eq:normal_line}
	is given by 
\begin{align}
	\label{eq:line_dist_2d}
	d = \frac{\abs{   \vec{n}^{\top}\vec{P}-c }}{\norm{\vec{n}}}	
\end{align}
		\solution Without loss of generality, let $\vec{A}$ be the foot of the perpendicular from $\vec{P}$ to the line in 
	\eqref{eq:dir_line}.  The equation of the normal to 
	\eqref{eq:normal_line} can then be expressed as 
\begin{align}
	\label{eq:dir_line_normal_dist}
	\vec{x} &= \vec{A} + \lambda \vec{n}
	\\
	\implies 
	\vec{P}- \vec{A} &=  \lambda \vec{n}
	\label{eq:dir_line_normal_dist_pa}
\end{align}
$\because \vec{P}$ lies on 
		\eqref{eq:dir_line_normal_dist}.
From the above, the desired distance can be expressed as 
\begin{align}
d = 	\norm{\vec{P}- \vec{A}}= \abs{\lambda} \norm{\vec{n}}
	\label{eq:dir_line_normal_dist_pa_d}
\end{align}
From 
	\eqref{eq:dir_line_normal_dist_pa},
\begin{align}
	\vec{n}^{\top}
	\brak{\vec{P}- \vec{A}} &=  \lambda \vec{n}^{\top}\vec{n} = \lambda\norm{\vec{n}}^2
	\\
	\implies \abs{\lambda}&= \frac{\abs{\vec{n}^{\top}
	\brak{\vec{P}- \vec{A}}}}{\norm{\vec{n}}^2} 
\end{align}
	Substituting the above in \eqref{eq:dir_line_normal_dist_pa_d} and using 
	the fact that 
\begin{align}
   \vec{n}^{\top}\vec{A} = c
\end{align}
from 	\eqref{eq:normal_line}, yields 
	\eqref{eq:line_dist_2d}.

	\item The distance from the origin to the line  in 
	\eqref{eq:normal_line}
	is given by 
\begin{align}
	\label{eq:dist_line_2d_orig}
	d = \frac{\abs{   c }}{\norm{\vec{n}}}	
\end{align}
\item The distance between the parallel lines 
\begin{align}
	\label{eq:parallel_lines}
	\begin{split}
		\vec{n}^{\top}\vec{x} &= c_1
		\\
		\vec{n}^{\top}\vec{x} &= c_2
	\end{split}
\end{align}
is given by 
\begin{align}
	\label{eq:dist_lines_2d}
	d = \frac{\abs{   c_1-c_2 }}{\norm{\vec{n}}}	
\end{align}
\item The equation of the line perpendicular to 
	\eqref{eq:normal_line}
		and passing through the point $\vec{P}$ is given by 
\begin{align}
	\vec{m}^{\top}\brak{\vec{x}-\vec{P}}  = 0
\end{align}
\item The foot of the perpendicular from $\vec{P}$ to the line in 
	\eqref{eq:normal_line}
	is given by 
\begin{align}
	\label{eq:normal_line_foot}
	\myvec{ \vec{m} & \vec{n}}^{\top}\vec{x}= \myvec{\vec{m}^{\top}\vec{P}\\ c }  
\end{align}
% 
\solution From
	\eqref{eq:normal_line} and 
	\eqref{eq:normal_line_pt} 
the foot of the perpendicular satisfies the equations 
\begin{align}
	\vec{n}^{\top}\vec{x} &= c
	\\
	\vec{m}^{\top}\brak{\vec{x}-\vec{P} }&=0 
\end{align}
where $\vec{m}$ is the direction vector of the given line.  Combining the above into a matrix equation results in 
	\eqref{eq:normal_line_foot}.
\end{enumerate}
 
\subsection{Three Dimensions}
\renewcommand{\theequation}{\theenumi}
%\begin{enumerate}[label=\arabic*.,ref=\theenumi]
\begin{enumerate}[label=\thesubsection.\arabic*.,ref=\thesubsection.\theenumi]
\numberwithin{equation}{enumi}
\item The area of a triangle with vertices $\vec{A}, \vec{B}, \vec{C}$ is given by 
\begin{align}
  \label{eq:area3d}
 \frac{1}{2} \norm{\brak{\vec{A} - \vec{B}} \times \brak{\vec{A} - \vec{C}}}
\end{align}

\item Points $\vec{A}, \vec{B}, \vec{C}$ are on a line if 
\begin{align}
  \label{eq:line_rank}
  \text{rank}\myvec{\vec{A} \\ \vec{B} \\ \vec{C} }  = 1
\end{align}
\item Points $\vec{A}, \vec{B}, \vec{C}, \vec{D}$ form a paralelogram if 
\begin{align}
  \label{eq:parallelgm_rank}
  \text{rank}\myvec{\vec{A} \\ \vec{B} \\ \vec{C} \\ \vec{D}  }  = 1, 
  \text{rank}\myvec{\vec{A} \\ \vec{B} \\ \vec{C} }  = 2
\end{align}
\item The equation of a line  is given by  
	\eqref{eq:dir_line}
	\item The equation of a plane is given by
	\eqref{eq:normal_line}
	\item The distance from the origin to the line  in 
	\eqref{eq:normal_line}
	is given by 
	\eqref{eq:dist_line_2d_orig}
\item The distance from a point $\vec{P}$  to the line in 
	\eqref{eq:dir_line} is given by 
\begin{align}
	\label{dist_3d_def_final}
		d = \norm{\vec{A} -\vec{P}}^2 - \frac{\cbrak{\vec{m}^{\top}\brak{\vec{A}-\vec{P} 
	}}^2}{\norm{\vec{m}}^2}
%	d =\norm{\vec{A}  -\vec{P}
% -\frac{\vec{m}^{\top}\brak{\vec{A} 
%			-\vec{P}}}
%			{ \norm{\vec{m}}^2}
%	\vec{m}}
		\end{align}
		\solution 
\begin{align}
	\label{dist_3d_def}
	d\brak{\lambda } &=\norm{\vec{A} + \lambda \vec{m}-\vec{P}}
	\\
\implies 	d^2\brak{\lambda } &=\norm{\vec{A} + \lambda \vec{m}-\vec{P}}^2
\end{align}
which can be simplified to obtain 
	\begin{multline}
d^2\brak{\lambda } =\lambda^2 \norm{\vec{m}}^2+2\lambda \vec{m}^{\top}\brak{\vec{A} 
		-\vec{P}}
		\\
		+\norm{\vec{A} -\vec{P}}^2
	\end{multline}
which is of the form 
\begin{align}
	\label{dist_3d_def_quad}
	d^2\brak{\lambda } &=a \lambda^2 + 2b\lambda +c
	\\
	&=a \cbrak{\brak{\lambda+ \frac{b}{a}}^2 +\sbrak{\frac{c}{a}-\brak{\frac{b}{a}}^2 }}
\end{align}
with 
\begin{align}
	\label{dist_3d_def_quad_abc}
	a = \norm{\vec{m}}^2, b = \vec{m}^{\top}\brak{\vec{A} 
		-\vec{P}}, c = 
		\norm{\vec{A} -\vec{P}}^2
\end{align}
which can be expressed as 
%		\begin{multline}
%			d^2\brak{\lambda } =\norm{\vec{m}}^2\brak{\lambda + \frac{\vec{m}^{\top}\brak{\vec{A}-\vec{P} }}{\vec{m}}^2}}^2 +2\lambda \vec{m}^{\top}\brak{\vec{A} 
%			-\vec{P}}
%			\\
%			+\norm{\vec{A} -\vec{P}}^2
%		\end{multline}
		From the above, $d^2\brak{\lambda}$ is smallest when upon substituting from 
	\eqref{dist_3d_def_quad_abc}
\begin{align}
	\label{dist_3d_def_quad_small}
	\lambda+ \frac{b}{2a} &= 0 \implies \lambda = - \frac{b}{2a}
	\\
	&= -\frac{\vec{m}^{\top}\brak{\vec{A} 
			-\vec{P}}}
			{ \norm{\vec{m}}^2}
	%		\label{dist_3d_lam}
\end{align}
and consequently, 
\begin{align}
	d_{\min}\brak{\lambda } &=a \brak{\frac{c}{a}-\brak{\frac{b}{a}}^2 } 
	\\
	&=c - \frac{b^2}{a }
\end{align}
yielding
	\eqref{dist_3d_def_final} after substituting from 
	\eqref{dist_3d_def_quad_abc}.
%From 	\eqref{dist_3d_def} and \eqref{dist_3d_lam}, 
%	\eqref{dist_3d_def} is obtained.
\item The distance between the parallel planes 
	\eqref{eq:parallel_lines}
	is given by 
	\eqref{eq:dist_lines_2d}.
\item The plane 
		\begin{align}
		\vec{n}^{\top}
			\vec{x} = c
			\label{eq:plain_contain}
		\end{align}
		contains the line 
		\begin{align}
			\vec{x} = \vec{A}+\lambda \vec{m}
			\label{eq:line_contain}
		\end{align}
		if 
		\begin{align}
		\vec{m}^{\top}\vec{n} = 0
			\label{eq:line_plain_contain}
		\end{align}
		\solution Any point on the line 
			\eqref{eq:line_contain}
			should also satisfy 
			\eqref{eq:plain_contain}.  Hence, 
		\begin{align}
			\vec{n}^{\top}\brak{\vec{A}+\lambda \vec{m}} &= \vec{n}^{\top}\vec{A}=c
		\end{align}
		which can be simplified to obtain
			\eqref{eq:line_plain_contain}
		\item The foot of the perpendicular from a point $\vec{P}$ to the plane 
		\begin{align}
			\vec{n}^{\top}\vec{x} =c
		\end{align}
		is given by 
		\\
		\solution The equation of the line perpendicular to the given plane and passing through $\vec{P}$ is 
		\begin{align}
			\vec{x} = \vec{P} + \lambda 	\vec{n}
		\end{align}
		From 
	\eqref{eq:dir_line_plane_isect}, the intersection of the above line with the given plane is 
\begin{align}
	\vec{x} &= \vec{P} + \frac{c - \vec{n}^{\top}\vec{P}}{\norm{\vec{n}}^2}
\vec{n}
	\label{eq:foot_perp_pt_plane}
\end{align}
\item The image of a point $\vec{P}$ with respect to the plane 
		\begin{align}
			\vec{n}^{\top}\vec{x} =c
		\end{align}
		is given by 
		\begin{align}
			\vec{R} &=
	  \vec{P} + 2\frac{c - \vec{n}^{\top}\vec{P}}{\norm{\vec{n}}^2}
			\label{eq:image_pt_plane}
		\end{align}
		\solution Let $\vec{R}$ be the desired image.  Then, subtituting the expression for the  foot of the perpendicular from $\vec{P}$ to the given plane using 
	\eqref{eq:foot_perp_pt_plane},
		\begin{align}
			\frac{\vec{P}+\vec{R}}{2} &=
	  \vec{P} + \frac{c - \vec{n}^{\top}\vec{P}}{\norm{\vec{n}}^2}
		\end{align}
		\item Let a plane pass through the points $\vec{A},\vec{B}$ and be perpendicular to the plane 
		\begin{align}
		\vec{n}^{\top}\vec{x} =c 
			\label{eq:plane_3d_2pt_perp_given}
		\end{align}
		Then the equation of this plane is given by 
		\begin{align}
		\vec{p}^{\top}\vec{x} = 1
			\label{eq:plane_3d_2pt}
		\end{align}
		where
		\begin{align}
			\vec{p} = 		\myvec{\vec{A} & \vec{B} & \vec{n}}^{-\top}  \myvec{1 \\ 1 \\ 0}
			\label{eq:plane_3d_2pt_perp_norm}
		\end{align}
	\solution From the given information, 
		\begin{align}
			\vec{p}^{\top}\vec{A} &=d 
			\\
			\vec{p}^{\top}\vec{B} &=d 
			\\
			\vec{p}^{\top}\vec{n} &= 0
			\label{eq:plane_3d_2pt_perp_system}
		\end{align}
		$\because$ the normal vectors to the two planes will also be perpendicular.  The system of equations in 
			\eqref{eq:plane_3d_2pt_perp_system}
			can be expressed as the matrix equation
		\begin{align}
			\myvec{\vec{A} & \vec{B} & \vec{n}}^{\top}\vec{p} = d\myvec{1 \\ 1 \\ 0}
			\label{eq:plane_3d_2pt_perp_system_temp}
		\end{align}
		which yields 
			\eqref{eq:plane_3d_2pt_perp_norm}
			upon normalising with $d$.
		\item The intersection of the line represented by 
	\eqref{eq:dir_line}
	with the plane represented by 
	\eqref{eq:normal_line}
	is given by 
\begin{align}
	\label{eq:dir_line_plane_isect}
	\vec{x} &= \vec{A} + \frac{c - \vec{n}^{\top}\vec{A}}{\vec{n}^{\top}\vec{m}}
\vec{m}
\end{align}
\solution From 
	\eqref{eq:dir_line}
	and 
	\eqref{eq:normal_line},
\begin{align}
	\vec{x} &= \vec{A} + \lambda \vec{m}
	\\
	\vec{n}^{\top}\vec{x} &= c
	\\
	\implies 
	\vec{n}^{\top}\brak{\vec{A} + \lambda \vec{m}}&= c
	\label{eq:dir_line_plane_inter}
\end{align}
which can be simplified to obtain
\begin{align}
	\vec{n}^{\top}\vec{A} + \lambda 	\vec{n}^{\top}\vec{m}&= c
	\\
	\implies \lambda &= \frac{c - \vec{n}^{\top}\vec{A}}{\vec{n}^{\top}\vec{m}}
\end{align}
Substituting the above in 
	\eqref{eq:dir_line_plane_inter}
	yields
	\eqref{eq:dir_line_plane_isect}.
\item The foot of the perpendicular from the point $\vec{P}$ to the line  represented by 
	\eqref{eq:dir_line}
	is given by 
\begin{align}
	\label{eq:plane_line_foot_ans}
	\vec{x} &= \vec{A} + \frac{ \vec{m}^{\top}\brak{\vec{P} - \vec{A}}}{\norm{\vec{m}}^2}
\vec{m}
\end{align}
\solution  Let the equation of the line be 
\begin{align}
	\label{eq:dir_line_foot}
	\vec{x} &= \vec{A} + \lambda \vec{m}
\end{align}
	The equation of the plane perpendicular to the given line passing through $\vec{P}$ is given by
\begin{align}
	\label{eq:plane_line_foot}
	\vec{m}^{\top}\brak{\vec{x}-\vec{P}}  &= 0
	\\
	\implies \vec{m}^{\top}\vec{x}  &= \vec{m}^{\top}\vec{P}
\end{align}
The desired foot of the perpendicular is the intersection of 
	\eqref{eq:dir_line_foot} with 
	\eqref{eq:plane_line_foot}
	which can be obtained from 
	\eqref{eq:dir_line_plane_isect}
	as 
	\eqref{eq:plane_line_foot_ans}
\item The foot of the perpendicular from a point $\vec{P}$ to a plane is $\vec{Q}$.  The equation of the plane is given by 
\begin{align}
	\label{eq:plane_foot_perp}
	\brak{\vec{P}-\vec{Q}}^{\top}\brak{\vec{x}-\vec{Q}} = 0
\end{align}
	\solution  The normal vector to the plane is given by 
\begin{align}
	\vec{n}= \vec{P}-\vec{Q} 
\end{align}
	Hence, the equation of the plane is
	\eqref{eq:plane_foot_perp}.
\item Let $\vec{A}, \vec{B}, \vec{C}$ be  points on a plane.  The equation of the plane is then given by 	
\begin{align}
	\myvec{	\vec{A} & \vec{B}& \vec{C}}^{\top} \vec{n}= \myvec{1\\1\\1}
	\label{eq:plane_3pt}
\end{align}
\solution Let the equation of the plane be 
\begin{align}
	\vec{n}^{\top}	\vec{x} &= 1
\end{align}
Then 
\begin{align}
	\vec{n}^{\top}	\vec{A} &= 1
	\\
	\vec{n}^{\top}	\vec{B} &= 1
	\\
	\vec{n}^{\top}	\vec{C} &= 1
\end{align}
which can be combined to obtain 
	\eqref{eq:plane_3pt}.
%\renewcommand{\theequation}{\theenumi}
%%\begin{enumerate}[label=\arabic*.,ref=\theenumi]
%\begin{enumerate}[label=\thesubsection.\arabic*.,ref=\thesubsection.\theenumi]
%\numberwithin{equation}{enumi}
%
\item (Parallelogram Law)  Let $\vec{A}, \vec{B}, \vec{D}$ be three vertices of a parallelogram.  Then the vertex $\vec{C}$ is given by 
\begin{align}
  \label{eq:pgm_law}
  \vec{C} = \vec{B}+\vec{C} - \vec{A}
\end{align}
		\solution Shifting $\vec{A}$ to the origin, we obtain a parallelogram with corresponding vertices 
\begin{align}
  \label{eq:pgm_law_org_vert}
  \vec{0}, \vec{B}-\vec{A}, \vec{D} - \vec{A}
\end{align}
The fourth vertex of this parallelogram is then obtained as 
\begin{align}
  \label{eq:pgm_law_org}
	\brak{\vec{B}-\vec{A}}+\brak{ \vec{D} - \vec{A}} = \vec{D}+ \vec{B} - 2\vec{A}
\end{align}
Shifting the origin to $\vec{A}$, the fourth vertex is obtained as 
\begin{align}
  \label{eq:pgm_law_org_C}
		 \vec{C} &= \vec{D}+ \vec{B} - 2\vec{A}+\vec{A} 
		 \\
	 &=
	 \vec{D}+ \vec{B} - \vec{A} 
\end{align}
\item (Affine Transformation) Let $\vec{A},\vec{C}$, be opposite vertices of a square. The other two points can be obtained as  
\begin{align}
  \label{eq:square_points}
  \vec{B} = \frac{\norm{\vec{A}-\vec{C}}}{\sqrt{2}} \vec{P}\vec{e}_1+\vec{A}
  \\
  \vec{D} = \frac{\norm{\vec{A}-\vec{C}}}{\sqrt{2}} \vec{P}\vec{e}_2+\vec{A}
\end{align}
where 
\begin{align}
	\vec{P} = \myvec{\cos \brak{\theta-\frac{\pi}{4}} & \sin  \brak{\theta-\frac{\pi}{4}} \\ \sin \brak{\theta-\frac{\pi}{4}} & \cos \brak{\theta-\frac{\pi}{4}}}
\end{align}
and 
\begin{align}
	\cos\theta = \frac{\brak{\vec{C}-\vec{A}}^{\top}\vec{e}_1}{\norm{\vec{A}-\vec{C}}\norm{\vec{e}_1}}
\end{align}
\end{enumerate}
\section{Quadratic Forms}
\subsection{Conic Sections}
\renewcommand{\theequation}{\theenumi}
%\begin{enumerate}[label=\arabic*.,ref=\theenumi]
\begin{enumerate}[label=\thesubsection.\arabic*.,ref=\thesubsection.\theenumi]
\numberwithin{equation}{enumi}
\item Let $\vec{P}$ be a point such that the ratio of its distance from a fixed point $\vec{F}$ and the distance ($d$) from a fixed line 
$L: \vec{n}^{\top}\vec{x}=c$ is constant, given by 
\begin{align}
\label{conics/30/def}
\frac{\norm{\vec{P}-\vec{F}}}{d} = e    
\end{align}
The locus of $\vec{P}$ such is known as a conic section. The line $L$ is known as the directrix and the point $\vec{F}$ is the focus. $e$ is defined to be 
the eccentricity of the conic.  
\begin{enumerate}
    \item For $e = 1$, the conic is a parabola
    \item For $e < 1$, the conic is an ellipse
    \item For $e > 1$, the conic is a hyperbola
\end{enumerate}
\item The equation of  a conic with directrix $\vec{n}^{\top}\vec{x} = c$, eccentricity $e$ and focus $\vec{F}$ is given by 
\begin{align}
    \label{eq:conic_quad_form}
    \vec{x}^{\top}\vec{V}\vec{x}+2\vec{u}^{\top}\vec{x}+f=0
    \end{align}
where     
\begin{align}
  \label{eq:conic_quad_form_v}
\vec{V} &=\norm{\vec{n}}^2\vec{I}-e^2\vec{n}\vec{n}^{\top}, 
\\
\label{eq:conic_quad_form_u}
\vec{u} &= ce^2\vec{n}-\norm{\vec{n}}^2\vec{F}, 
\\
\label{eq:conic_quad_form_f}
f &= \norm{\vec{n}}^2\norm{\vec{F}}^2-c^2e^2
%\\
    \end{align}
    \solution  From \eqref{conics/30/def} and \eqref{eq:line_dist_2d},  for any point $\vec{x}$ on the conic,
		\begin{align}	
			\norm{\vec{x}-\vec{F}}^2=e^2 \frac{\brak{{\vec{n}^{\top}\vec{x} - c}}^2}{\norm{\vec{n}}^2}\label{conics/30/eq:1} \\
			\implies \norm{\vec{n}}^2\brak{\vec{x}-\vec{F}}^{\top}\brak{\vec{x}-\vec{F}}=e^2\brak{\vec{n}^{\top}\vec{x} - c}^2
    \end{align}
    yielding
\begin{multline}
\norm{\vec{n}}^2\brak{\vec{x}^{\top}\vec{x}-2\vec{F}^{\top}\vec{x}+\norm{\vec{F}}^2}
	\\
	=e^2\brak{c^2+\brak{\vec{n}^{\top}\vec{x} }^2-2c\vec{n}^{\top}\vec{x}} 
	\\
=e^2\brak{c^2+\brak{\vec{x}^{\top}\vec{n}\vec{n}^{\top}\vec{x} }-2c\vec{n}^{\top}\vec{x}}
% t\vec{x}^{\top}\vec{x}-(\vec{n}^{\top}\vec{x} )^2-2t\vec{F}^{\top}\vec{x}+2c\vec{n}^{\top}\vec{x}=c^2-t\norm{\vec{F}}^2\\
% t\vec{x}^{\top}\vec{I}\vec{x}-\vec{n}^{\top}\vec{x} \vec{n}^{\top}\vec{x}+2(c\vec{n}-t\vec{F})^{\top}\vec{x}=c^2-t\norm{\vec{F}}^2\\
% \vec{x}^{\top}(t\vec{I}-\vec{n}\vec{n}^{\top})\vec{x}+2(c\vec{n}-t\vec{F})^{\top}\vec{x}+t\norm{\vec{F}}^2-c^2=0
\end{multline}
%
which can be expressed as \eqref{eq:conic_quad_form} after simplification.
\item \eqref{eq:conic_quad_form} represents 
	\begin{enumerate}
		\item a parabola for $\mydet{\vec{V}} = 0 $,
		\item ellipse for $\mydet{\vec{V}} > 0 $ and 
		\item hyperbola for $\mydet{\vec{V}} < 0 $.
	\end{enumerate}
			In general
\eqref{eq:conic_quad_form}  represents a conic  if and only if
\begin{align}
\mydet{
\vec{V}&\vec{u}
\\
\vec{u}^{\top}&f
}
\ne  0
\label{eq:quad_forms_pair_det}
\end{align}
%
else, it represents a pair of straight lines.
\end{enumerate}
\subsection{Conic Parameters}
\renewcommand{\theequation}{\theenumi}
%\begin{enumerate}[label=\arabic*.,ref=\theenumi]
\begin{enumerate}[label=\thesubsection.\arabic*.,ref=\thesubsection.\theenumi]
\numberwithin{equation}{enumi}
\item The conic in     \eqref{eq:conic_quad_form} can be expressed in standard form (centre/vertex at the origin, major axis - $x$ axis) as
  \begin{align}
    %\begin{aligned}
    \label{eq:conic_simp_temp_nonparab}
    \vec{y}^{\top}\vec{D}\vec{y} &=  \vec{u}^{\top}\vec{V}^{-1}\vec{u} -f  &  \abs{V} &\ne 0
    \\
    \vec{y}^{\top}\vec{D}\vec{y} &=  -2\eta\vec{e}_1^{\top}\vec{y}   & \abs{V} &= 0
    \label{eq:conic_simp_temp_parab}
    %\end{aligned}
    \end{align}

    where
    \begin{align}
      %\begin{split}
      \label{eq:conic_parmas_eig_def}
      \vec{P}^{\top}\vec{V}\vec{P} &= \vec{D}. \quad \text{(Eigenvalue Decomposition)}
      \\
      \vec{D} &= \myvec{\lambda_1 & 0\\ 0 & \lambda_2}, 
      \\
      \vec{P} &= \myvec{\vec{p}_1 & \vec{p}_2}, \quad \vec{P}^{\top}=\vec{P}^{-1},
      \label{eq:eigevecP}
      \\
      \label{eq:eta}
       \eta &=\vec{u}^{\top}\vec{p}_1
       \\
       \vec{e}_1 &=\myvec{1 \\ 0}
      \end{align}
      \solution Using 
\begin{align}
\vec{x} = \vec{P}\vec{y}+\vec{c} \quad \text{(Affine Transformation)}
\label{eq:conic_affine}
\end{align}
%such that 
\eqref{eq:conic_quad_form} can be expressed as

%\item  
%Substituting \eqref{eq:conic_affine} in \eqref{eq:conic_quad_form}
\begin{multline}
\brak{\vec{P}\vec{y}+\vec{c}}^T\vec{V}\brak{\vec{P}\vec{y}+\vec{c}}+2\vec{u}^T\brak{\vec{P}\vec{y}+\vec{c}}+ f
	\\
	= 0, 
\end{multline}
yielding 
\begin{multline}
\vec{y}^T\vec{P}^T\vec{V}\vec{P}\vec{y}+2\brak{\vec{V}\vec{c}+\vec{u}}^T\vec{P}\vec{y}
\\
+  \vec{c}^T\vec{V}\vec{c} + 2\vec{u}^T\vec{c} + f= 0
\label{eq:conic_simp_one}
\end{multline}
%
From \eqref{eq:conic_simp_one} and \eqref{eq:conic_parmas_eig_def},
\begin{multline}
\vec{y}^T\vec{D}\vec{y}+2\brak{\vec{V}\vec{c}+\vec{u}}^T\vec{P}\vec{y}
\\
+  \vec{c}^T\brak{\vec{V}\vec{c} + \vec{u}}+ \vec{u}^T\vec{c} + f= 0
\label{eq:conic_simp}
\end{multline}
When $\vec{V}^{-1}$ exists,
\begin{align}
%\begin{split}
\vec{V}\vec{c}+\vec{u} &= \vec{0}, \quad \text{or}, \vec{c} = -\vec{V}^{-1}\vec{u},
\label{eq:conic_parmas_c_def}
\end{align}
%
%%From \eqref{eq:conic_parmas_k_def} and 
%%
and substituting \eqref{eq:conic_parmas_c_def}
in \eqref{eq:conic_simp}
yields \eqref{eq:conic_simp_temp_nonparab}. 
%\item  
When $\abs{\vec{V}} = 0, \lambda_1 = 0$ and 
\begin{align}
\vec{V}\vec{p}_1 = 0, 
\vec{V}\vec{p}_2 = \lambda_2\vec{p}_2.
\label{eq:conic_parab_eig_prop} 
\end{align}
where $\vec{p}_1,\vec{p}_2$ are the eigenvectors of $\vec{V}$ such that  \eqref{eq:conic_parmas_eig_def}
%
\begin{align}
\vec{P} = \myvec{\vec{p}_1 & \vec{p}_2},
\label{eq:eig_matrix}
\end{align}
Substituting \eqref{eq:eig_matrix}
in \eqref{eq:conic_simp},
\begin{multline}
	\vec{y}^T\vec{D}\vec{y}+2\brak{\vec{c}^T\vec{V}+\vec{u}^T}\myvec{\vec{p}_1 & \vec{p}_2}\vec{y}
\\
+  \vec{c}^T\brak{\vec{V}\vec{c} + \vec{u}}+ \vec{u}^T\vec{c} + f= 0
\\
\implies \vec{y}^T\vec{D}\vec{y}
\\
+2\myvec{\brak{\vec{c}^T\vec{V}+\vec{u}^T}\vec{p}_1  \brak{\vec{c}^T\vec{V}+\vec{u}^T}\vec{p}_2}\vec{y}
\\
+  \vec{c}^T\brak{\vec{V}\vec{c} + \vec{u}}+ \vec{u}^T\vec{c} + f= 0
\\
\implies \vec{y}^T\vec{D}\vec{y}
\\
+2\myvec{\vec{u}^T\vec{p}_1 & \brak{\lambda_2\vec{c}^T+\vec{u}^T}\vec{p}_2}\vec{y}
\\
+  \vec{c}^T\brak{\vec{V}\vec{c} + \vec{u}}+ \vec{u}^T\vec{c} + f= 0
\text{ from } \eqref{eq:conic_parab_eig_prop}     \nonumber \\
\\
\implies \lambda_2y_2^2+2\brak{\vec{u}^T\vec{p}_1}y_1+  2y_2\brak{\lambda_2\vec{c}+\vec{u}}^T\vec{p}_2
\\
+  \vec{c}^T\brak{\vec{V}\vec{c} + \vec{u}}+ \vec{u}^T\vec{c} + f= 0
\label{eq:conic_parab_foc_len_temp} 
\end{multline}
which is the equation of a parabola. 
Thus, \eqref{eq:conic_parab_foc_len_temp} 
can be expressed as \eqref{eq:conic_simp_temp_parab} by choosing
\begin{align}
%\label{eq:eta}
\eta = \vec{u}^T\vec{p}_1
\end{align}
%Choosing 
%\begin{align}
%\vec{u} + \lambda_2\vec{c} = 0,
%\vec{c}^T\brak{\vec{V}\vec{c} + \vec{u}}+ \vec{u}^T\vec{c} + f = 0,
%\end{align}
% the above equation becomes
%\begin{align}
%y_2^2= -\frac{2\vec{u}^T\vec{p}_1}{ \lambda_2} \brak{y_1
%+  \frac{\vec{u}^T\vec{V}\vec{u} - 2\lambda_2\vec{u}^T\vec{u} + f\lambda_2^2}{2\vec{u}^T\vec{p}_1\lambda_2^2}}
%\\
%or \eta = 2\vec{u}^T\vec{p}_1
%%\label{eq:conic_simp_parab_new}
%\end{align}
and $\vec{c}$ in \eqref{eq:conic_simp} such that
\begin{align}
\label{eq:conic_parab_one}
\vec{P}^{T}\brak{\vec{V}\vec{c}+\vec{u}} &= \eta\myvec{1\\0}
\\
\vec{c}^T\brak{\vec{V}\vec{c} + \vec{u}}+ \vec{u}^T\vec{c} + f&= 0
\label{eq:conic_parab_two}
\end{align}
%we obtain  \eqref{eq:conic_simp_temp_parab}.
Multiplying \eqref{eq:conic_parab_one} by $\vec{P}$ yields
\begin{align}
\label{eq:conic_parab_one_eig}
\brak{\vec{V}\vec{c}+\vec{u}} &= \eta\vec{p}_1,
\end{align}
which, upon substituting in \eqref{eq:conic_parab_two}
results in 
\begin{align}
\eta\vec{c}^T\vec{p}_1 + \vec{u}^T\vec{c} + f&= 0
\label{eq:conic_parab_two_eig}
\end{align}
\eqref{eq:conic_parab_one_eig} and \eqref{eq:conic_parab_two_eig} can be clubbed together to obtain \eqref{eq:conic_parab_c}.

\item The centre/vertex of the conic is given by 
  \begin{align}
    %\begin{aligned}[t]
    \label{eq:conic_nonparab_c}
    \vec{c} &= - \vec{V}^{-1}\vec{u} & \abs{V} &\ne 0
    \\
    \myvec{ \vec{u}^{\top}+\eta\vec{p}_1^{\top} \\ \vec{V}}\vec{c} &= \myvec{-f \\ \eta\vec{p}_1-\vec{u}}  &\abs{V} &= 0
    %\end{cases}
    %\end{aligned}
    \label{eq:conic_parab_c}
    \end{align}      
    \solution From \eqref{eq:conic_affine},
\begin{align}
\label{eq:conic_affine_inv}
\vec{y} = \vec{P}^{\top}\brak{\vec{x}-\vec{c}}
\end{align}
For the standard conic, $\vec{y} = \vec{0} $ is the centre/vertex and in \eqref{eq:conic_affine_inv}, 
\begin{align}
\label{eq:conic_centre}
\vec{y} = \vec{0} \implies \vec{x}=\vec{c}
\end{align}
\item The focal length of the parabola in \eqref{eq:conic_simp_temp_parab} is given by 
  \begin{align}
    \mydet{\frac{2\eta}{\lambda_2}} 
    %= \mydet{\frac{2\vec{u}^T\vec{p}_1}{\lambda_2}}.
    \label{eq:conic_parab_foc_len} 
    \end{align}    
    where $\lambda_2$ is the nonzero eigenvalue of $\vec{V}$ and $\eta$ is defined in \eqref{eq:eta}.
    % \begin{align}
    %   \eta = \vec{u}^T\vec{p}_1
    %   \end{align}
    \item For $\mydet{V} \ne 0$, the lengths of the semi-major and semi-minor axes of the conic in \eqref{eq:conic_quad_form} are given by 
  \begin{align} 
    \label{eq:ellipse_axes}
  %  \begin{aligned}[t]
    \sqrt{\frac{\vec{u}^{\top}\vec{V}^{-1}\vec{u} -f}{\lambda_1}}, 
    \sqrt{\frac{\vec{u}^{\top}\vec{V}^{-1}\vec{u} -f}{\lambda_2}}. \quad \brak{\text{ellipse}}
    \\
%
       \sqrt{\frac{\vec{u}^{\top}\vec{V}^{-1}\vec{u} -f}{\lambda_1}}, 
       \sqrt{\frac{f-\vec{u}^{\top}\vec{V}^{-1}\vec{u}}{\lambda_2}}, \quad \brak{\text{hyperbola }}
%
  %\end{aligned}
  \label{eq:hyper_axes}
\end{align} 
\solution For \begin{align} \abs{\vec{V}} > 0, \quad \text{or, } \lambda_1 > 0, \lambda_2 > 0 
  \end{align} and \eqref{eq:conic_simp_temp_nonparab} becomes \begin{align} \lambda_1y_1^2 +\lambda_2y_2^2 = 
  \vec{u}^{\top}\vec{V}^{-1}\vec{u} -f \end{align} yielding        \eqref{eq:ellipse_axes}.  Similarly, \eqref{eq:hyper_axes} can be obtained for 
  \begin{align} 
    \label{eq:conic_hyper_cond}
    \abs{\vec{V}} 
    < 0, \quad \text{or, } \lambda_1 > 0, \lambda_2 < 0 \end{align}
    \item The equation of the minor and major  axes are respectively given by 
  \begin{align}
\vec{p}_i^{\top}\brak{\vec{x}-\vec{c}} = 0, i = 1,2
  \end{align}
  \item The eccentricity, directrices and foci of \eqref{eq:conic_quad_form} are given by 
  \eqref{eq:conic_quad_form_e} -
  \eqref{eq:conic_quad_form_F} 
  \begin{figure*}[!hb]
	  \centering
	  \hrule
\begin{align}
  \label{eq:conic_quad_form_e} 
  e&= \sqrt{1-\frac{\lambda_1}{\lambda_2}}
\\
\label{eq:conic_quad_form_nc} 
  \vec{n}&= \sqrt{\lambda_2}\vec{p}_1,  
  \\
	c &= 
  \begin{cases}
    \frac{e\vec{u}^{\top}\vec{n} \pm \sqrt{e^2\brak{\vec{u}^{\top}\vec{n}}^2-\lambda_2\brak{e^2-1}\brak{\norm{\vec{u}}^2 - \lambda_2 f}}}{\lambda_2e\brak{e^2-1}} & e \ne 1
    \\
    \frac{\norm{\vec{u}}^2 - \lambda_2 f   }{2e^2\vec{u}^{\top}\vec{n}} & e = 1
  \end{cases}
  \\
  \label{eq:conic_quad_form_F} 
  \vec{F}  &= \frac{ce^2\vec{n}-\vec{u}}{\lambda_2}
\end{align}  
  \end{figure*}
\solution 
From \eqref{eq:conic_quad_form_v},
  \begin{multline}
    \vec{V}^{\top} \vec{V}=\brak{\norm{\vec{n}}^2\vec{I}-e^2\vec{n}\vec{n}^{\top}}^{\top}
	  \\
	  \brak{\norm{\vec{n}}^2\vec{I}-e^2\vec{n}\vec{n}^{\top}}
    \\
    \implies \vec{V}^{2} = \norm{\vec{n}}^4\vec{I}+e^4\vec{n}\vec{n}^{\top}\vec{n}\vec{n}^{\top}
	  \\
	  -2e^2\norm{\vec{n}}^2\vec{n}\vec{n}^{\top}
    \\
    = \norm{\vec{n}}^4\vec{I} + e^4\norm{\vec{n}}^2\vec{n}\vec{n}^{\top}
	%  \\
	  - 2e^2\norm{\vec{n}}^2\vec{n}\vec{n}^{\top}
    \\
    = \norm{\vec{n}}^4\vec{I} + e^2\brak{e^2 - 2}\norm{\vec{n}}^2\vec{n}\vec{n}^{\top}
    \\
    = \norm{\vec{n}}^4\vec{I} + \brak{e^2 - 2}\norm{\vec{n}}^2\brak{\norm{\vec{n}}^2\vec{I}- \vec{V}}
    \end{multline}
%    
which can be expressed as
\begin{align}
  \vec{V}^{2} + \brak{e^2 - 2}\norm{\vec{n}}^2\vec{V} - \brak{e^2 - 1}\norm{\vec{n}}^4\vec{I}=0
  \label{eq:conic_quad_form_e_cayley}
\end{align}
Using the Cayley-Hamilton theorem, \eqref{eq:conic_quad_form_e_cayley} results in the characteristic equation, 
\begin{align}
  \lambda^{2} - \brak{2-e^2}\norm{\vec{n}}^2\lambda + \brak{1-e^2 }\norm{\vec{n}}^4=0
\end{align}
which can be expressed as
\begin{multline}
\brak{\frac{\lambda}{\norm{\vec{n}}^2}}^2 - \brak{2-e^2 }\brak{\frac{\lambda}{\norm{\vec{n}}^2}} 
	\\
	+ \brak{1-e^2 } = 0
\end{multline}
\begin{align}
  \implies \frac{\lambda}{\norm{\vec{n}}^2} = 1-e^2, 1
  \\
  \text{or, }\lambda_2 = \norm{\vec{n}}^2, \lambda_1 = \brak{1-e^2}\lambda_2 
  \label{eq:conic_quad_form_lam_cayley}
\end{align}
From   \eqref{eq:conic_quad_form_lam_cayley}, the eccentricity of \eqref{eq:conic_quad_form} is given by 
\eqref{eq:conic_quad_form_e}.   
%
% By inspection, we find that 
% \begin{align}
%   \frac{\lambda}}{\norm{\vec{n}}^2} = 1
%   \label{eq:conic_quad_form_lam2_cayley}
% \end{align}
%satisfies \eqref{eq:conic_quad_form_lam_cayley}.
Multiplying both sides of    \eqref{eq:conic_quad_form_v} by $\vec{n}$,
\begin{align}
\vec{V} \vec{n}&=\norm{\vec{n}}^2\vec{n}-e^2\vec{n}\vec{n}^{\top}\vec{n} 
\\
&=\norm{\vec{n}}^2\brak{1-e^2}\vec{n} 
 \\
% &=\frac{\lambda_1}{\lambda_2}\norm{\vec{n}}^2\vec{n} 
% \end{align}
% upon substituting from \eqref{eq:conic_quad_form_e}  and simplifying.  From the above, it is obvious that $\vec{n}$ is an eigenvector
% of $\vec{V}$.  Choosing 
% \begin{align}
%   \lambda_2 = \norm{\vec{n}}^2,
%   \label{eq:eigevecn_lam2}
% \end{align}  
% we obtain 
% \begin{align}
  &=\lambda_1 \vec{n} 
  \label{eq:eigevecn}
\end{align}  
from \eqref{eq:conic_quad_form_lam_cayley}
Thus,  $\lambda_1$ is the corresponding eigenvalue for $\vec{n}$.  From       \eqref{eq:eigevecP},   \eqref{eq:conic_quad_form_lam_cayley} and \eqref{eq:eigevecn}, 
\begin{align}
   \vec{n}&= \norm{\vec{n}}\vec{p}_1  = \sqrt{\lambda_2}\vec{p}_1 
\end{align}  
From \eqref{eq:conic_quad_form_u} and \eqref{eq:conic_quad_form_lam_cayley},
\begin{align}
%   \label{eq:conic_quad_form_v}
% \vec{V} &=\norm{\vec{n}}^2\vec{I}-e^2\vec{n}\vec{n}^{\top}, 
% \\
%\label{eq:conic_quad_form_u}
\vec{F}  &= \frac{ce^2\vec{n}-\vec{u}}{\lambda_2}
 \\
 \implies \norm{\vec{F}}^2  &= \frac{\brak{ce^2\vec{n}-\vec{u}}^{\top}\brak{ce^2\vec{n}-\vec{u}}}{\lambda_2^2}
 \\
 \implies \lambda_2^2\norm{\vec{F}}^2  &= c^2e^4\lambda_2-2ce^2\vec{u}^{\top}\vec{n}+\norm{\vec{u}}^2
 \label{eq:conic_quad_form_u_temp}
% f &= \norm{\vec{n}}^2\norm{\vec{F}}^2-c^2e^2
% %\\
    \end{align}
    Also, \eqref{eq:conic_quad_form_f} can be expressed as
    \begin{align}
    \lambda_2\norm{\vec{F}}^2 &= f+c^2e^2
    \label{eq:conic_quad_form_f_temp}
\end{align}
From  \eqref{eq:conic_quad_form_u_temp} and     \eqref{eq:conic_quad_form_f_temp},
\begin{align}
c^2e^4\lambda_2-2ce^2\vec{u}^{\top}\vec{n}+\norm{\vec{u}}^2 = \lambda_2\brak{f+c^2e^2}
\end{align}
\begin{multline}
\implies \lambda_2e^2\brak{e^2-1}c^2-2ce^2\vec{u}^{\top}\vec{n}
	\\
	+\norm{\vec{u}}^2 - \lambda_2 f = 0
\end{multline}
yielding
  \eqref{eq:conic_quad_form_F}. 
%\begin{align}
%\text{or, } c = 
%\begin{cases}
%  \frac{e\vec{u}^{\top}\vec{n} \pm \sqrt{e^2\brak{\vec{u}^{\top}\vec{n}}^2-\lambda_2\brak{e^2-1}\brak{\norm{\vec{u}}^2 - \lambda_2 f}}}{\lambda_2e\brak{e^2-1}} & e \ne 1
%  \\
%  \frac{\norm{\vec{u}}^2 - \lambda_2 f   }{2e^2\vec{u}^{\top}\vec{n}} & e = 1
%\end{cases}
%\end{align}
  \end{enumerate}
\subsection{Tangent and Normal}
\renewcommand{\theequation}{\theenumi}
%\begin{enumerate}[label=\arabic*.,ref=\theenumi]
\begin{enumerate}[label=\thesubsection.\arabic*.,ref=\thesubsection.\theenumi]
\numberwithin{equation}{enumi}
\item The points of intersection of the line 
\begin{align}
L: \quad \vec{x} = \vec{q} + \mu \vec{m} \quad \mu \in \mathbb{R}
\label{eq:conic_tangent}
\end{align}
with the conic section in \eqref{eq:conic_quad_form} are given by
\begin{align}
\vec{x}_i = \vec{q} + \mu_i \vec{m}
\label{eq:conic_tangent_pts}
\end{align}
%
where
{\tiny
\begin{multline}
\mu_i = \frac{1}
{
\vec{m}^T\vec{V}\vec{m}
}
\lbrak{-\vec{m}^T\brak{\vec{V}\vec{q}+\vec{u}}}
\\
\pm
\rbrak{\sqrt{
\sbrak{
\vec{m}^T\brak{\vec{V}\vec{q}+\vec{u}}
}^2
-
\brak
{
\vec{q}^T\vec{V}\vec{q} + 2\vec{u}^T\vec{q} +f
}
\brak{\vec{m}^T\vec{V}\vec{m}}
}
}
\label{eq:tangent_roots}
\end{multline}
}
\\
\solution 
Substituting \eqref{eq:conic_tangent}
in \eqref{eq:conic_quad_form}, 
\begin{align}
\brak{\vec{q} + \mu \vec{m}}^T\vec{V}\brak{\vec{q} + \mu \vec{m}}  
\\
+ 2 \vec{u}^T\brak{\vec{q} + \mu \vec{m}}+f &= 0
\\
\implies \mu^2\vec{m}^T\vec{V}\vec{m} + 2 \mu\vec{m}^T\brak{\vec{V}\vec{q}+\vec{u}} 
\\
+ \vec{q}^T\vec{V}\vec{q} + 2\vec{u}^T\vec{q} +f &= 0
\label{eq:conic_intercept}
\end{align}
Solving the above quadratic in \eqref{eq:conic_intercept}
yields \eqref{eq:tangent_roots}.
  \item ({\em Latus Rectum}) The latus rectum of a conic section is the chord that passes through the focus, is perpendicular to the major axis and has both endpoints on the curve.
  \item The latus rectum is parallel to the directrix.
  \item The equation of the latus rectum is given by 
\begin{align}
	 \vec{n}^{\top}\brak{\vec{x}-\vec{F}} = 0
	\\
	\text{or, }\vec{x} = \vec{F} + \mu\vec{m}
\end{align}
where $\vec{F}$ is the focus and $\vec{m}$  is the normal to the directrix, i.e.
\begin{align}
	\vec{m}^{\top} \vec{n} = 0
\end{align}
\item The affine transform preserves the norm.  This implies that the length of any chord of a conic
	is invariant to translation and/or rotation.
	\solution Let 
From \eqref{eq:conic_affine}, 
\begin{align}
\vec{x}_i = \vec{P}\vec{y}_i+\vec{c} 
\label{eq:conic_affine_pts}
\end{align}
be any two points on the conic.  Then the distance between the points is given by 
\begin{align}
	\norm{\vec{x}_1-\vec{x}_2 } &= \norm{\vec{P}	\vec{y}_1 -\vec{y}_2 }
	\\
	\implies \norm{\vec{x}_1-\vec{x}_2 }^2 &= 		\brak{\vec{y}_1 -\vec{y}_2 }^{\top}\vec{P}^{\top}\vec{P}\brak{\vec{y}_1 -\vec{y}_2 }
	\\
	&= 		\norm{\vec{y}_1 -\vec{y}_2 }^2
\label{eq:conic_affine_norm_preserve}
\end{align}
since 
\begin{align}
	\vec{P}^{\top}\vec{P} = \vec{I}
\end{align}
\item The length of the latus rectum is given by 
	\solution 
\\
From 
\eqref{eq:conic_tangent_pts}
and
\eqref{eq:tangent_roots}, substituting $\vec{q} = \vec{F}$, the end points of the latus rectum on the conic
section can be obtained.  Thus, the distance between these points is given by 
\begin{align}
	\norm{\vec{x}_1-\vec{x}_2} =  \abs{\mu_1-\mu_2} \norm{\vec{m}}
\label{eq:conic_tangent_pts_dist}
\end{align}
which can be used to obtain the length of the latus rectum as 
	{\tiny
\begin{multline}
 \frac{2\sqrt{
\sbrak{
\vec{m}^T\brak{\vec{V}\vec{F}+\vec{u}}
}^2
-
\brak
{
\vec{F}^T\vec{V}\vec{F} + 2\vec{u}^T\vec{F} +f
}
\brak{\vec{m}^T\vec{V}\vec{m}}
}
}
{
\vec{m}^T\vec{V}\vec{m}
}\norm{\vec{m}}
\label{eq:tangent_roots_latus}
\end{multline}
}
\begin{enumerate}
\item 
From \eqref{eq:conic_affine_norm_preserve}, we may consider the standard ellipse/hyperbola given by 
    \eqref{eq:conic_simp_temp_nonparab} as
  \begin{multline}
    \vec{y}^{\top}\vec{D}\vec{y} = -f,
	\vec{V} =     \vec{D} = \myvec{\lambda_1 & 0 \\ 0 &\lambda_2},
	  \\
	   f = -1, \vec{u} = 0, \vec{p}_1 = \vec{e}_1, \vec{p}_2 = \vec{e}_2
	    \label{eq:latus_rectum_ellipse_param}
\end{multline}
for computing the length of the latus rectum in 
	  \eqref{eq:tangent_roots_latus}. Note that $\vec{p}_1, \vec{p}_2$ are the eigenvectors and $\vec{e}_1, \vec{e}_2$ are the standard basis vectors.  Substituting from 
	    \eqref{eq:latus_rectum_ellipse_param} in
\eqref{eq:conic_quad_form_nc},
	     the parameters of the directrix are obtained as 

\begin{align}
\label{eq:conic_quad_form_nc_latus} 
  \vec{n}&= \sqrt{\lambda_2}\vec{e}_1,  
  \\
	c &=	\pm   \frac{ 1 }{e\sqrt{e^2-1} }
\end{align}  
and the focus is 
\begin{align}
  \label{eq:conic_quad_form_e_latus} 
  e&= \sqrt{1-\frac{\lambda_1}{\lambda_2}}
  \\
  \label{eq:conic_quad_form_F_latus} 
	\vec{F}  &= \frac{ e  }{\sqrt{\lambda_2}\sqrt{e^2-1}}\vec{e}_1
\end{align}  
From \eqref{eq:conic_quad_form_nc_latus},  
\begin{align}  
	\vec{m} = \vec{e}_1
\end{align}  
Substituting the above in 
\eqref{eq:tangent_roots_latus} along with
  \eqref{eq:conic_quad_form_e_latus} and  
  \eqref{eq:conic_quad_form_F_latus}, 


	the length of the latus rectum for an ellipse and hyperbola is obtained from 
\eqref{eq:tangent_roots_latus} as .
	{\tiny
\begin{multline}
 \frac{2\sqrt{
\sbrak{
\vec{m}^T\brak{\vec{V}\vec{F}+\vec{u}}
}^2
-
\brak
{
\vec{F}^T\vec{V}\vec{F} + 2\vec{u}^T\vec{F} +f
}
\brak{\vec{m}^T\vec{V}\vec{m}}
}
}
{
\vec{m}^T\vec{V}\vec{m}
}\norm{\vec{m}}
%\label{eq:tangent_roots_latus}
\end{multline}
}

	\solution For simplicity, we consider the standard ellipse given by 
\begin{align}
	\vec{x}^{\top}\vec{V}\vec{x} &= 1,
	\\
	\text{where }
	\vec{V} = \myvec{\lambda_1 & 0 \\ 0 & \lambda_2}, \vec{u} = 0, f &= -1
  \label{eq:conic_quad_form_std_ellipse}, 
\end{align}
From 
  \eqref{eq:conic_quad_form_std_ellipse} and  
  \eqref{eq:conic_quad_form_F}, 
  the distance between the foci can be expressed as
\begin{align}
  \label{eq:conic_quad_form_std_ellipse_foci_dir}, 
	\norm{\vec{F}_1 - \vec{F}_2}  &= e^2\abs{\frac{c_1-c_2}{\lambda_2}}\norm{\vec{n}}
\end{align}
%
where $c_1,c_2$ are the scalar parameters of the directrices.  The distances between the directrices is given by 
\begin{align}
	\frac{\abs{c_1-c_2}}{\norm{\vec{n}}}
\end{align}
Thus, substituting the above in 
  \eqref{eq:conic_quad_form_std_ellipse_foci_dir}, 
\begin{align}
  \label{eq:conic_quad_form_std_ellipse_foci_dir_ratio}, 
	\frac{\norm{\vec{F}_1 - \vec{F}_2}}{\abs{c_1-c_2}} \norm{n} &= e^2\frac{\norm{\vec{n}}^2}{\abs{\lambda_2}}
	\\
	&=\frac{6}{12} = \frac{1}{2},
\end{align}
based on the given information.  For the standard ellipse, 
\begin{align}
	\vec{p}_1 &= \vec{e}_1 = \myvec{1 \\ 0}
	\\
	\implies \vec{n} &= \sqrt{\lambda_2}\vec{e}_1,  
	\\
	\text{or, }\vec{m} &=\sqrt{\lambda_2}\vec{e}_2,   
\end{align}
Hence, substituting in 
  \eqref{eq:conic_quad_form_std_ellipse_foci_dir_ratio} and using  
\eqref{eq:conic_quad_form_nc} 
\begin{align}
	e^2\frac{\norm{\vec{n}}^2}{\lambda_2} &= \frac{1}{2}
	\\
	\implies e^2 = \frac{1}{2}
\end{align}
Substituting $\vec{u} = 0$ in 
\eqref{eq:conic_quad_form_nc} and 
  \eqref{eq:conic_quad_form_F} 
  and simplifying using
  \eqref{eq:conic_quad_form_e},
\begin{align}
	c &= \pm \frac{1}{e}\sqrt{\frac{\lambda_1}{\lambda_2}}
	\\
	\vec{F} &= \pm \frac{e}{\sqrt{\lambda_1}}\vec{p}_1
\end{align}
For the standard ellipse, $\vec{m}$, orthogonal to $\vec{n}$ is also an eigenvector, such that 
\begin{align}
	\vec{V}\vec{m} &= {\lambda_1}\vec{m}  
	\\
	\implies\vec{m} &=  \sqrt{\lambda_1}\vec{e}_2
\end{align}
Thus, 
\begin{align}
	\vec{m}^{\top}\vec{V}\vec{m} &= {\lambda_1}^2
	\\
	\vec{m}^{\top}\vec{V}\vec{F} &= 0
	\\
	\vec{F}^{\top}\vec{V}\vec{F} &=e^2 
\label{eq:tangent_roots_latus_quad},
\end{align}
From 
\eqref{eq:tangent_roots_latus},
the length of the latus rectum is given by 
{\tiny
\begin{multline}
 \frac{2\sqrt{
\sbrak{
\vec{m}^T\brak{\vec{V}\vec{F}+\vec{u}}
}^2
-
\brak
{
\vec{F}^T\vec{V}\vec{F} + 2\vec{u}^T\vec{F} +f
}
\brak{\vec{m}^T\vec{V}\vec{m}}
}
}
{
\vec{m}^T\vec{V}\vec{m}
}\norm{\vec{m}}
\end{multline}
}
Substituting \eqref{eq:tangent_roots_latus_quad} in the above, the desired length can be expressed as,
%\eqref{eq:conic_quad_form_nc}, 
%{\tiny
%\begin{multline}
% \frac{2\sqrt{
%1-e^2
%	}}
%\lambda_1^2
%	}
%}
%{
%\lambda_1^2
%	}\sqrt{\lambda_1}
%\end{multline}
\end{enumerate}

\item  If $L$ in \eqref{eq:conic_tangent} touches \eqref{eq:conic_quad_form} at exactly one point $\vec{q}$, 
  \begin{align}
  \vec{m}^T\brak{\vec{V}\vec{q}+\vec{u}} = 0
  \label{eq:conic_tangent_mq}
  \end{align}
\\
\solution
In this case, \eqref{eq:conic_intercept} has exactly one root.  Hence, 
  in \eqref{eq:tangent_roots}
  \begin{multline}
  \sbrak{
  \vec{m}^T\brak{\vec{V}\vec{q}+\vec{u}}
  }^2 -\brak{\vec{m}^T\vec{V}\vec{m}}
  \\
  \brak
  {
  \vec{q}^T\vec{V}\vec{q} + 2\vec{u}^T\vec{q} +f
  } = 0                                                                                             
  \label{eq:conic_tangent_disc}
  \end{multline}                    
  $\because \vec{q}$ is the point of contact, $\vec{q}$ satisfies \eqref{eq:conic_quad_form}
  and 
  \begin{align}
  \vec{q}^T\vec{V}\vec{q} + 2\vec{u}^T\vec{q} +f = 0
  \label{eq:conic_tangent_qquad}
  \end{align}
  Substituting \eqref{eq:conic_tangent_qquad} in \eqref{eq:conic_tangent_disc} and simplifying, we obtain \eqref{eq:conic_tangent_mq}.
  \item Given the point of contact $\vec{q}$, the equation of a tangent to \eqref{eq:conic_quad_form} is 
  \begin{align}
  \brak{\vec{V}\vec{q}+\vec{u}}^T\vec{x}+\vec{u}^T\vec{q}+f = 0
  \label{eq:conic_tangent_final}
  \end{align}
  \solution The normal vector is obtained from \eqref{eq:conic_tangent_mq} 
and \eqref{eq:normal_line}
  as
  %
  \begin{align}
  \label{eq:conic_normal_vec}
  \vec{n} = \vec{V}\vec{q}+\vec{u}
  \end{align}  
  From \eqref{eq:conic_normal_vec} and \eqref{eq:normal_line_pt}, the equation of the tangent is\begin{align}
    \brak{\vec{V}\vec{q}+\vec{u}}^T\brak{\vec{x}-\vec{q}} &=0
    \\
    \implies \brak{\vec{V}\vec{q}+\vec{u}}^T\vec{x}-\vec{q}^T\vec{V}\vec{q}-\vec{u}^T\vec{q} &= 0
    \end{align}
    which, upon substituting from \eqref{eq:conic_tangent_qquad} and simplifying yields \eqref{eq:conic_tangent}.
    \item   If $\vec{V}^{-1}$ exists, given the normal vector $\vec{n}$, the tangent points of contact to \eqref{eq:conic_quad_form} are given by
\begin{align}
  \begin{split}
\vec{q}_i &= \vec{V}^{-1}\brak{\kappa_i \vec{n}-\vec{u}}, i = 1,2
\\
\text{where }\kappa_i &= \pm \sqrt{
\frac{
\vec{u}^T\vec{V}^{-1}\vec{u}-f
}
{
\vec{n}^T\vec{V}^{-1}\vec{n}
}
}
  \end{split}
\label{eq:conic_tangent_qk}
\end{align}
\solution From \eqref{eq:conic_normal_vec},
\begin{align}
\label{eq:conic_normal_vec_q}
 \vec{q} = \vec{V}^{-1}\brak{\kappa \vec{n}-\vec{u}}, \quad \kappa \in \mathbb{R}
\end{align}
Substituting \eqref{eq:conic_normal_vec_q}
in \eqref{eq:conic_tangent_qquad},
\begin{multline}
\brak{\kappa \vec{n}-\vec{u}}^T\vec{V}^{-1}\brak{\kappa \vec{n}-\vec{u}} 
\\
+ 2\vec{u}^T\vec{V}^{-1}\brak{\kappa \vec{n}-\vec{u}} +f = 0
\end{multline}
\begin{align}
\implies 
\kappa^2 \vec{n}^T\vec{V}^{-1}\vec{n} - \vec{u}^T\vec{V}^{-1}\vec{u} + f &=0
 \\
 \text{or, } \kappa = \pm \sqrt{\frac{\vec{u}^T\vec{V}^{-1}\vec{u}-f}{\vec{n}^T\vec{V}^{-1}\vec{n}}} &\label{eq:conic_normal_k}
\end{align}
%
%yileding 
Substituting \eqref{eq:conic_normal_k} in \eqref{eq:conic_normal_vec_q}
yields \eqref{eq:conic_tangent_qk}.
\item If $\vec{V}$ is not invertible,  given the normal vector $\vec{n}$, the point of contact to \eqref{eq:conic_quad_form} is given by the matrix equation
\begin{align}
\label{eq:conic_tangent_q_eigen}
\begin{pmatrix}
	\vec{\vec{u}^{\top}+\kappa \vec{n}}^{\top} \\ \vec{V}
\end{pmatrix}
\vec{q} &= 
\begin{pmatrix}
-f
\\
\kappa\vec{n}-\vec{u}
\end{pmatrix}
\\
\text{where }  \kappa = \frac{\vec{p}_1^T\vec{u}}{\vec{p}_1^T\vec{n}}, \quad \vec{V}\vec{p}_1 &= 0
\label{eq:conic_tangent_qk_eigen}
\end{align}
\solution 
  If $\vec{V}$ is non-invertible, it has a zero eigenvalue.  If the corresponding eigenvector is $\vec{p}_1$, then,
\begin{align}
\vec{V}\vec{p}_1 = 0
\label{eq:conic_zero_eigen}
\end{align}
From \eqref{eq:conic_normal_vec},
\begin{align}
\label{eq:conic_zero_eigen_normal}
\kappa \vec{n} &= \vec{V} \vec{q}+\vec{u}, \quad \kappa \in \mathbb{R}
\\
\implies \kappa \vec{p}_1^T\vec{n} &= \vec{p}_1^T\vec{V} \vec{q}+\vec{p}_1^T\vec{u}
\\
\text{or, } \kappa \vec{p}_1^T\vec{n} &= \vec{p}_1^T\vec{u},  \quad \because \vec{p}_1^T \vec{V} = 0, 
\\
%\quad 
	&\brak{\text{ from } \eqref{eq:conic_zero_eigen}}
%\label{eq:conic_normal_vec_q}
\end{align}
yielding $\kappa$ in \eqref{eq:conic_tangent_qk_eigen}. From \eqref{eq:conic_zero_eigen_normal},
\begin{align}
\kappa \vec{q}^T\vec{n} &= \vec{q}^T\vec{V} \vec{q}+\vec{q}^T\vec{u}
\\
\implies \kappa \vec{q}^T\vec{n} &= -f-\vec{q}^T\vec{u} \quad \text{from } \eqref{eq:conic_tangent_qquad},
\\
	\text{or, } \brak{\kappa \vec{n}+\vec{u}}^{\top}\vec{q} &= -f
\label{eq:conic_zero_eigen_normal_fq}
\end{align}
\eqref{eq:conic_zero_eigen_normal} can be expressed as
\begin{align}
\label{eq:conic_zero_eigen_normal_vq}
\vec{V} \vec{q} = \kappa \vec{n} - \vec{u}.
\end{align}
\eqref{eq:conic_zero_eigen_normal_fq} and \eqref{eq:conic_zero_eigen_normal_vq} clubbed together result in \eqref{eq:conic_tangent_q_eigen}.
\item When \eqref{eq:conic_quad_form} is a hyperbola, its  {\em asymptotes}  are defined as the pair of intersecting straight lines 
  \begin{align}
  \label{eq:asymp_quad_form}
  \vec{x}^{\top}\vec{V}\vec{x}+2\vec{u}^{\top}\vec{x}+\vec{u}^{\top}\vec{V}^{-1}\vec{u}=0, \quad \abs{\vec{V}} < 0
  \end{align}
%  such that 
  % \begin{align} 
  % %\label{eq:quad_form_asymp_cond}
  % %K =  \vec{u}^{\top}\vec{V}^{-1}\vec{u}
  % %\\
  
  % \label{eq:quad_pair_det}
  % \end{align} 
\item \eqref{eq:asymp_quad_form} can be expressed as the lines 
%
\begin{align} 
\label{eq:quad_form_pair}
\myvec{\sqrt{\abs{\lambda_1}} & \pm \sqrt{\abs{\lambda_2}}}\vec{P}^{\top}\brak{\vec{x}-\vec{c}} = 0
\end{align} 
\solution Reducing   \eqref{eq:asymp_quad_form} to standard form using the {\em affine transformation} yields
  \begin{align} 
    \lambda_1y_1^2 -\brak{-\lambda_2}y_1^2 = 0
    \label{eq:quad_form_hyper}
    \end{align}
    From \eqref{eq:asymp_quad_form},
    %
    %\eqref{eq:quad_form_hyper} and \eqref{eq:quad_form_asymp_cond} 
    the equation of the asymptotes for \eqref{eq:quad_form_hyper} is
    \begin{align} 
    \myvec{\sqrt{\abs{\lambda_1}} & \pm \sqrt{\abs{\lambda_2}}}\vec{y} = 0
    \end{align} 
  from which \eqref{eq:quad_form_pair} is obtained using \eqref{eq:conic_affine}.
  % \begin{align} 
  %   \label{eq:quad_form_pair}
  %   \myvec{\sqrt{\abs{\lambda_1}} & \pm \sqrt{\abs{\lambda_2}}}\vec{P}^T\brak{\vec{x}-\vec{c}} = 0
  %   \end{align} 
  \item The angle between the asymptotes is then given by using the inner product
\begin{align} 
\label{eq:quad_form_pair_ang}
\cos\theta=\frac{\abs{\lambda_1}-\abs{\lambda_2}}
{\abs{\lambda_1}+\abs{\lambda_2}}
\end{align} 
\item The normal vectors of the lines in \eqref{eq:quad_form_pair} are 
  \begin{align} 
  \label{eq:quad_form_pair_normvecs}
  \begin{split}
  \vec{n}_1 &= \vec{P}\myvec{\sqrt{\abs{\lambda_1}} \\[2mm]  \sqrt{\abs{\lambda_2}}}
  \\
  \vec{n}_2 &= \vec{P}\myvec{\sqrt{\abs{\lambda_1}} \\[2mm] - \sqrt{\abs{\lambda_2}}}
  \end{split}
  \end{align} 
  The angle between the asymptotes is given by 
\begin{align} 
\label{eq:quad_form_pair_ang_exp}
\cos\theta=\frac{\vec{n_1}^{\top}\vec{n_2}}{\norm{\vec{n_1}}\norm{\vec{n_2}}}
\end{align} 
The orthogonal matrix $\vec{P}$ preserves the norm, i.e.
\begin{align} 
\norm{\vec{n_1}} = \norm{\vec{P}\myvec{\sqrt{\abs{\lambda_1}} \\[2mm]  \sqrt{\abs{\lambda_2}}}}
\\
=\norm{\myvec{\sqrt{\abs{\lambda_1}} \\[2mm]  \sqrt{\abs{\lambda_2}}}}
=\sqrt{\abs{\lambda_1}+\abs{\lambda_2}} = \norm{\vec{n_2}}
\end{align} 
It is easy to verify that 
\begin{align} 
\vec{n_1}^{\top}\vec{n_2} = \abs{\lambda_1}-\abs{\lambda_2}
\end{align} 
%
Thus, the angle between the asymptotes is obtained from \eqref{eq:quad_form_pair_ang_exp} as \eqref{eq:quad_form_pair_ang}.
% \begin{align} 
% \label{eq:quad_form_pair_ang}
% \cos\theta=\frac{\abs{\lambda_1}-\abs{\lambda_2}}
% {\abs{\lambda_1}+\abs{\lambda_2}}
% \end{align} 
\item Another hyperbola with the same asymptotes as \eqref{eq:quad_form_pair} can be obtained from \eqref{eq:conic_quad_form} and \eqref{eq:asymp_quad_form} as
\begin{align}
\label{eq:hyper_conj_quad_form}
\vec{x}^{\top}\vec{V}\vec{x}+2\vec{u}^{\top}\vec{x}+2\vec{u}^{\top}\vec{V}^{-1}\vec{u}-f=0
\end{align}
\end{enumerate}
\subsection{Intersection of Conics}
\renewcommand{\theequation}{\theenumi}
%\begin{enumerate}[label=\arabic*.,ref=\theenumi]
\begin{enumerate}[label=\thesubsection.\arabic*.,ref=\thesubsection.\theenumi]
\numberwithin{equation}{enumi}
\item Let the intersection of the conics
\begin{align}
    \label{eq:conic_quad_form_ints}
    \vec{x}^{\top}\vec{V}_i\vec{x}+2\vec{u}_i^{\top}\vec{x}+f_i=0, \quad i = 1,2
    \end{align}
    be 
\begin{multline}
    \label{eq:conic_quad_form_int}
	\vec{x}^{\top}\brak{\vec{V}_1+ \mu \vec{V}_2}\vec{x}+2\brak{\vec{u}_1+\mu \vec{u}_2}^{\top}\vec{x}
	\\
	+f_1+\mu f_2=0
    \end{multline}
From \eqref{eq:quad_forms_pair_det}, the above equation will represent a pair of straight lines if and only if 
\begin{align}
\mydet{
\vec{V}_1+ \mu \vec{V}_2&\vec{u}_1+\mu\vec{u}_2
\\
\vec{u}_1+\mu\vec{u}_2}^{\top}&f
}
\ne  0
\end{align}
which can be expressed as 
\begin{multline}
\mydet{
\vec{V}_1+ \mu \vec{V}_2&\vec{u}_1+\mu\vec{u}_2
\\
\vec{u}_1+\mu\vec{u}_2}^{\top}&f
}
\ne  0
\end{multline}
\end{enumerate}
\end{document}


